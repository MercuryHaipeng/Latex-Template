% ---------- 导言开始 ----------
% 1. 版面
% \usepackage[a4paper,inner=1.5cm, outer=3cm, top=2cm, bottom=3cm,bindingoffset=5mm]{geometry}
\usepackage[a4paper,margin=2.5cm]{geometry} % 统一边距2.5cm
\usepackage[protrusion=true, expansion=false, final]{microtype} % 字体微调
\usepackage[english]{babel}     % 美式 → May 1, 2025
% \usepackage[british]{babel}   % 英式 → 1st May 2025
% \usepackage[UTF8]{ctex}       % 中文 → 2025年5月1日
\usepackage{blindtext}   % 生成示例文本


% 2. 字体和段落
\usepackage{newpxtext,newpxmath}    % 字体,newpxtext 是 Palatino 字体,newpxmath 是配套的数学字体
\usepackage[onehalfspacing]{setspace}   % 1.5 倍行距
\usepackage{parskip}        % 段落首行无缩进,间距分隔
% Latex 默认标题后段落不缩进(section, subsection等)
% \usepackage{indentfirst}    % 设置首段落首行缩进


% 3. 图表和代码
\usepackage{graphicx}       % 插入图片
\usepackage{subcaption}     % 提供 subfigure 环境,用于子图
\graphicspath{{Figures/}}   % 图片搜索路径
\usepackage{diagbox}        % 生成斜线表头
\usepackage{listings}       % 行内代码引用 \lstinline||
\lstset{                    % 支持行内代码自动换行
  breaklines=true,
  breakatwhitespace=false
  columns=fullflexible
}
\usepackage{xcolor}     % 代码高亮
\lstdefinestyle{pythonstyle}{   % 定义 Python 代码风格
    language=Python,
    basicstyle=\ttfamily\small,
    keywordstyle=\color{blue},
    commentstyle=\color{teal},
    stringstyle=\color{orange},
    showstringspaces=false,
    numbers=left,
    numberstyle=\tiny,
    breaklines=true,
    frame=single,
    captionpos=b
}
\lstset{style=pythonstyle}


% 4. 公式和绘图
\usepackage{amsmath}    % 数学公式
\let\Bbbk\relax % 解决 amsfonts 和 newpxmath 的冲突
\usepackage{amssymb, amsfonts} % 数学符号和字体
\usepackage{tikz}    % 绘图
% 在物理学中,特别是经典力学和电磁学等领域,通常使用箭头来表示向量。
% \vec{} —— 短箭头,适合单个字母;\overrightarrow{} —— 长箭头,适合多字符。
% 在数学、工程学以及某些应用科学领域,更倾向于使用\bm{}表示向量或矩阵。这种方式避免了箭头可能带来的视觉混乱。
\usepackage{bm}        % 粗体数学符号


% 5. 交叉引用
% 交叉引用包的载入顺序如下,此载入顺序已经定义了如下命令:
% \newcommand{\cpageref}[1]{\vpageref{#1} on page~\pageref{#1}} % 带有“on page”的页码引用
% \vref{label} 中已经包含了 \vpageref{label},无需再调用 \vpageref
\usepackage[hidelinks]{hyperref}    % 生成可点击的目录, 图表等交叉引用, 需最先加载
\usepackage[nospace]{varioref}  % 智能页码引用(如“见第3页”),需在 hyperref 后加载
\usepackage[
    capitalize,     % \Cref 自动生成首字母大写
    nameinlink,     % 让“图1”整个成为可点击链接(而非仅数字)
    noabbrev        % 不使用缩写(如用 "Figure" 而不是 "Fig.")
]{cleveref}         % 智能引用(复数引用、范围引用等), 需在 hyperref 后加载
\usepackage{caption} % 自定义图表标题格式,见sty文件中的设置
% \crefname{} 命令可放入sty文件中
% 如果通过\href{run:}链接调用其他文件,路径和文件名称必须为英文,且相对路径是指对应的PDF文件,而不是tex文件!在'run:'后面没有空格。经测试,pdf和word均可以正常调用打开。


% 6. 参考文献
% 采用 BibLaTeX + Biber 方式,传统的 BibTeX 方式逐渐淘汰
\usepackage[backend=biber,style=ieee]{biblatex} % 参考文献,ieee 样式
\addbibresource{Bibliography/reference.bib}     % 参考文献数据库文件,可放入sty文件中
% 如果使用 BibTeX 而不是 Biber,可以使用以下两行代码
% \bibliographystyle{alpha} % 参考文献样式
% \bibliography{reference} % 参考文献数据库文件


% 7. 自定义命令
% 自定义命令,如\newcommand{cmd}{def}可以放在.sty文件中

% ---------- 导言结束 ----------